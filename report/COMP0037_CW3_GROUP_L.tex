\documentclass{article}


\usepackage{arxiv}

\usepackage{placeins}
\usepackage[utf8]{inputenc} % allow utf-8 input
\usepackage[T1]{fontenc}    % use 8-bit T1 fonts
\usepackage{hyperref}       % hyperlinks
\usepackage{url}            % simple URL typesetting
\usepackage{booktabs}       % professional-quality tables
\usepackage{amsfonts}       % blackboard math symbols
\usepackage{nicefrac}       % compact symbols for 1/2, etc.
\usepackage{microtype}      % microtypography

\usepackage{graphicx}	% to insert graphs
\usepackage{caption}	% to customize caption style
\usepackage{float}
\usepackage{subfigure}

\usepackage{amsmath}

\title{COMP0037 ASSIGNMENT 3}


\author{
 Group: \texttt{Group L}\\\\
 \textbf{Members:} \texttt{Yun Fang, Yusi Zhou}
}

%\date{}

\begin{document}

\maketitle

\captionsetup[figure]{labelformat={default},labelsep=period,name={Fig.}}

% -------------------------------------------------------------------------------------------
\section{Decision Re-Plan Policy}

Suppose B1 is the cell where the robot first detects the obstacle $O_{B}$. 

Suppose C1 is a cell on the aisle C at the same horizontal level as B1, as illustrated in the Figure 1-c on the assignment sheet. 

Let $T_{W}$ be the time the robot has to wait after the obstacle has been discovered.

Let c be the cost function associated with the path, where $c(L(\pi)) = \mathbb{E}(L(\pi)) $. \\

Note that:

$L_{XY}$ is the shortest path length between two cells X and Y, 

$T$ is the number of timesteps the obstacle remains in front of the robot where $ T = 0.5/ \lambda_{B} + \widetilde{T}$ with $\mathbb{E}(t) = 1/\lambda_{B}$ and $\mathbb{E}(\tilde{t}) = 0.5/ \lambda_{B}$ , 

$L_{W}$ is the cost of waiting a timestep.

% ------------------
\subsection{}

Let $\pi_{1}$ be the policy that the robot waits for the obstacle $O_{B}$ to clear.

Let $\pi_{2}$ be the policy that the robot plans a new path down aisle C.

Then,
\begin{align*}
c(L(\pi_{1})) &= c(L_{IB_{1}} + T_{W} \cdot L_{W} + L_{B_{1}B} + L_{BC} + L_{CG}) \\
c(L(\pi_{2})) &= c(L_{IB_{1}} +  L_{B_{1}C_{1}} + L_{C_{1}C} + L_{CG}) \\
\end{align*}

When, on average, the waiting policy is better than the other one: 
\begin{align*}
c(L(\pi_{1})) &\leq c(L(\pi_{2}))  \\
\mathbb{E}(L(\pi_{1})) &\leq \mathbb{E}(L(\pi_{2}))  \\
\mathbb{E}(L_{IB_{1}} + T_{W} \cdot L_{W} + L_{B_{1}B} + L_{BC} + L_{CG}) &\leq \mathbb{E}(L_{IB_{1}} +  L_{B_{1}C_{1}} + L_{C_{1}C} + L_{CG})  \\
\mathbb{E}(T_{W} \cdot L_{W} + L_{B_{1}B} + L_{BC}) &\leq \mathbb{E}(L_{B_{1}C_{1}} + L_{C_{1}C})  \\
\end{align*}

Because $L_{XY}$ is constant, $L_{W}$ is constant, and $L_{B_{1}B} = L_{C_{1}C}$:
\begin{align*}
\mathbb{E}(T_{W}) \cdot L_{W} + L_{BC} &\leq L_{B_{1}C_{1}}  \\
\mathbb{E}(T_{W}) \cdot L_{W} &\leq L_{B_{1}C_{1}} -  L_{BC}   \\
\mathbb{E}(T_{W}) &\leq \frac{L_{B_{1}C_{1}} -  L_{BC}}{L_{W} }   \\
\end{align*}

In this case, $T_{W} = 1 \cdot T = T$ , so:
\begin{align*}
\mathbb{E}(T) &\leq \frac{L_{B_{1}C_{1}} -  L_{BC}}{L_{W} }   \\
\frac{1}{\lambda_{B}} &\leq \frac{L_{B_{1}C_{1}} -  L_{BC}}{L_{W} }  \\
\lambda_{B} &\geq \frac{L_{W}} {L_{B_{1}C_{1}} -  L_{BC}} \\
\end{align*}

Therefore, the smallest value of $\lambda_{B}$ which guarantees on average that waiting is the better strategy is $\frac{L_{W}} {L_{B_{1}C_{1}} -  L_{BC}}$ and $\lambda_{B} \ne 0$ .

% ------------------
\subsection{}

Let $\pi_{0}$ be the policy that the robot drives directly down aisle B.

Let $\pi_{1}$ be the policy that the robot drives down aisle B, encounters an obstacle and waits.

Let $\pi_{2}$ be the policy that the robot drives down aisle B, encounters an obstacle, drives down aisle C.

Let $\pi_{3}$ be the policy that the robot drives directly down aisle C.

As $L(\pi_{0}) = L(\pi_{3})$ and thus $c(L(\pi_{0}) )= c(L(\pi_{3}))$, we will not discuss $\pi_{0}$ here. \\

Then,
\begin{align*}
c(L(\pi_{1})) &= c(L_{IB_{1}} + T_{W} \cdot L_{W} + L_{B_{1}B} + L_{BC} + L_{CG})\\
c(L(\pi_{2})) &= c(L_{IB_{1}} +  L_{B_{1}C_{1}} + L_{C_{1}C} + L_{CG}) \\
c(L(\pi_{3})) &= c(L_{IC_{1}} +  L_{C_{1}C} + L_{CG}) \\
\end{align*}

If the robot decides to drive directly down aisle C, then:
\begin{align}
c(L(\pi_{3})) &\leq c(L(\pi_{1})) \\
c(L(\pi_{3})) &\leq c(L(\pi_{2}))
\end{align}

by (1) - (2):
\begin{align*}
0 &\leq c(L(\pi_{1})) - c(L(\pi_{2})) \\
0 &\leq \mathbb{E}(L_{IB_{1}} + T_{W} \cdot L_{W} + L_{B_{1}B} + L_{BC} + L_{CG}) - \mathbb{E}(L_{IB_{1}} +  L_{B_{1}C_{1}} + L_{C_{1}C} + L_{CG}) \\
0 &\leq \mathbb{E} (T_{W}) \cdot L_{W} + L_{BC} - L_{B_{1}C_{1}} \\
\mathbb{E} (T_{W}) & \geq \frac{L_{B_{1}C_{1}} - L_{BC}} {L_{W}} \\
\mathbb{E} (T) & \geq \frac{L_{B_{1}C_{1}} - L_{BC}} {L_{W}} \\
\frac{1}{\lambda_{B}} & \geq \frac{L_{B_{1}C_{1}} - L_{BC}} {L_{W}} \\
\lambda_{B} & \leq \frac{L_{W}} {L_{B_{1}C_{1}} - L_{BC}} \\
\end{align*}

Therefore, the maximum value of $\lambda_{B}$ at which the robot will decide to drive directly down C and not attempt to drive down aisle B is $\frac{L_{W}} {L_{B_{1}C_{1}} - L_{BC}}$ and $\lambda_{B} \ne 0$ .

% ------------------
\subsection{}
In this case, $T_{W} = p_{B} \cdot T + (1 - p_{B}) \cdot 0 = p_{B} \cdot T$.

As $p_{B}$ is constant, $\mathbb{E}(T_{W}) = p_{B} \cdot \mathbb{E}(T) =  p_{B} / \lambda_{B}$. \\


Let $\pi_{1}$ be the policy that the robot drives down aisle B, encounters an obstacle and waits.

Let $\pi_{2}$ be the policy that the robot drives down aisle B, encounters an obstacle, drives down aisle C.

Let $\pi_{3}$ be the policy that the robot drives directly down aisle C.

Since in this situation the robot attempts to drive down aisle B first:
\begin{align}
c(L(\pi_{3})) &\geq c(L(\pi_{1})) \\
c(L(\pi_{3})) &\geq c(L(\pi_{2}))
\end{align}

Similarly as in the section 1.2, we can obtain that: 
\[
\mathbb{E} (T_{W}) \leq \frac{L_{B_{1}C_{1}} - L_{BC}} {L_{W}}
\]

As $\mathbb{E}(T_{W}) = p_{B} / \lambda_{B}$, $\lambda_{B}$ is a fixed value and $ \lambda_{B} \ne 0$,
\begin{align*}
p_{B} / \lambda_{B} &\leq \frac{L_{B_{1}C_{1}} - L_{BC}} {L_{W}} \\
p_{B}  &\leq \frac{\lambda_{B}(L_{B_{1}C_{1}} - L_{BC})} {L_{W}}
\end{align*}

Therefore, when $p_{B}$ is below $\frac{\lambda_{B}(L_{B_{1}C_{1}} - L_{BC})} {L_{W}}$, the robot will attempt to drive aisle B first.

% ------------------
\subsection{}

Let $\pi_{1}$ be the policy that the robot drives down aisle B and waits.

Let $\pi_{2}$ be the policy that the robot drives down aisle B, encounters an obstacle $O_{B}$, and drives down aisle C.

Let $\pi_{3}$ be the policy that the robot drives down aisle B, encounters an obstacle $O_{B}$, drives down aisle C, and waits.

Let $\pi_{4}$ be the policy that the robot drives down aisle B, encounters an obstacle $O_{B}$, drives down aisle C, encounters an obstacle $O_{C}$, and drives down aisle D.

Let $\pi_{5}$ be the policy that the robot drives directly down aisle D.

Suppose D1 is a cell on the aisle D at the same horizontal level as B1 and C1.

Then,
\begin{align*}
c(L(\pi_{1})) &= c(L_{IB_{1}} + T_{WB} \cdot L_{W} + L_{B_{1}B} + L_{BC} + L_{CG})\\
c(L(\pi_{2})) &= c(L_{IB_{1}} +  L_{B_{1}C_{1}} + L_{C_{1}C} + L_{CG}) \\
c(L(\pi_{3})) &= c(L_{IB_{1}} +  L_{B_{1}C_{1}} + T_{WC} \cdot L_{W}  + L_{C_{1}C} + L_{CG}) \\
c(L(\pi_{4})) &= c(L_{IB_{1}} +  L_{B_{1}C_{1}} +  L_{C_{1}D_{1}} + L_{D_{1}D} + L_{DG}) \\
c(L(\pi_{5})) &= c(L_{ID_{1}} +  L_{D_{1}D} + L_{DG}) \\
\end{align*}

And,
\begin{align*}
\mathbb{E}(T_{WB}) &=  p_{B} / \lambda_{B} \\
\mathbb{E}(T_{WC}) &=  p_{C} / \lambda_{C}
\end{align*}

We are looking to the path length, so if the robot drives directly down aisle D, it means:
\begin{align}
L(\pi_{5}) &\leq L(\pi_{1}) \\
L(\pi_{5}) &\leq L(\pi_{2}) \\
L(\pi_{5}) &\leq L(\pi_{3}) \\
L(\pi_{5}) &\leq L(\pi_{4}) \\
\end{align}

Sum up the four equation, we obtain that:
\begin{align*}
4 L(\pi_{5}) &\leq 4L_{IB_{1}}  + \mathbb{E}(T_{WB}) \cdot L_{W} +  \mathbb{E}(T_{WC}) \cdot L_{W}  
			+ 3 L_{B_{1}C_{1}} + L_{C_{1}D_{1}} + 4 L_{B_{1}B} 
			+ L_{BC} + 3L_{CG} + L_{DG} \\
4 L(\pi_{5}) &\leq 4L_{IB_{1}}  + (p_{B} / \lambda_{B}   +  p_{C} / \lambda_{C}) \cdot L_{W}  
			+ 3 L_{B_{1}C_{1}} + L_{C_{1}D_{1}} + 4 L_{B_{1}B} 
			+ L_{BC} + 3L_{CG} + L_{DG} \\
4 L(\pi_{5}) &\leq 4L_{IB_{1}}  + (p_{B} / \lambda_{B}   +  p_{C} / \lambda_{C}) \cdot L_{W}  
			+ 4 L_{B_{1}C_{1}} + 4 L_{B_{1}B} 
			+ 4L_{CG} \\
L(\pi_{5}) &\leq L_{IB_{1}}  + (p_{B} / \lambda_{B}   +  p_{C} / \lambda_{C}) \cdot 0.25L_{W}  
			+ L_{B_{1}C_{1}} + L_{B_{1}B} 
			+ L_{CG} \\
L(\pi_{5}) &\leq L_{IB_{1}}  + (p_{B} / \lambda_{B}   +  p_{C} / \lambda_{C}) \cdot 0.25L_{W}  
			+ L_{B_{1}C_{1}} + L_{C_{1}C} 
			+ L_{CG} \\
\end{align*}

Let L be the path length of $(L_{IB_{1}}  
			+ L_{B_{1}C_{1}} + L_{C_{1}C} 
			+ L_{CG} )$
			
Therefore, the upper bound on the value of the path length of the path going down aisle is $L+ 0.25L_{W}\cdot(p_{B} / \lambda_{B}   +  p_{C} / \lambda_{C})$.

\pagebreak
% -------------------------------------------------------------------------------------------
\section{Implement System in ROS}

\pagebreak
% -------------------------------------------------------------------------------------------

\bibliographystyle{unsrt}  
\begin{thebibliography}{1}


 \bibitem{WFD}
 Anirudh Topiwala; Pranav Inani; Abhishek Kathpal
\newblock (2018)
\newblock Frontier Based Exploration for Autonomous Robot
\newblock {\em<https://arxiv.org/abs/1806.03581>}.

 \bibitem{2}
Brian Yamauchi
\newblock (1997)
\newblock  A Frontier-Based Approach for Autonomous Exploration
\newblock {\em<https://www.semanticscholar.org/paper/A-frontier-based-approach-for-autonomous-Yamauchi/a1875055e9c526cbdc7abb161959d76d14b58266>}.

 \bibitem{InfoTheoretic}
Callum Rhodes; Cunjia Liu; Wen-Hua Chen
\newblock (2019)
\newblock An Information Theoretic Approach to Path Planning for Frontier Exploration
\newblock {\em<https://www.researchgate.net/publication/331929185\_An\_Information\_Theoretic\_Approach\_to\_Path\_Planning\_for\_Frontier\_Exploration>}.

 \bibitem{3}
Steven M. LaValle
\newblock (2006)
\newblock Planning Algorithm
\newblock {\em<http://planning.cs.uiuc.edu>}.

 \bibitem{4}
Matan Keidar; Gal A. Kaminka
\newblock Efficient Frontier Detection for Robot Exploration
\newblock Volume: 33 issue: 2
\newblock page(s):215-236
\newblock First published online: October 22, 2013
\newblock Issue published: February 1, 2014

 \bibitem{5}
Robert M. Gray
\newblock (2013)
\newblock Entropy and Information Theory
\newblock {\em<https://ee.stanford.edu/\~gray/it.pdf>}.



\end{thebibliography}


% -----------------------------------------------------------------------------------------
\end{document}